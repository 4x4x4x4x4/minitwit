\section{Reflection Perspective}

The biggest change to our workflow from before we gained this experience with DevOps, also representing our style of DevOps, is how an extensive pipeline empowers a developer to take more control of his own work. Usually, our group-work required a lot of checking in with the other developers, to make sure we are on the same page about how solutions are implemented, and ensuring we each agree on, and live up to the same quality and style standards. With an extensive pipeline, a lot of this can be alleviated by specific steps, that give a much stronger guarantee, and a stronger sense of confidence, that the code i am personally working on, is going to be fitting for the project, and the other developers, without waiting for, or maybe even interrupting, the other developers for confirmations. \\

An example of this are things like linting tools and static code analysis. These are tools that we use to inspect our code for problems regarding its style and maintainability. By building them into the pipeline, a developer will be notified of ways his code might breach the established standards, so he can fix them before merging code. This greatly reduces the need for code-reviews, which is a very concrete example, of how a developer more easily can stand by his own work all the way through delivery, without wasting time and energy by having another developer sign off on it. \\

This benefit is not limited to upholding good coding standards, but is as helpful in terms of sharing competencies. An issue we've experienced in earlier projects, is how specific skills accumulate with the developers that have worked the most with a specific area, which can make it difficult for developers to switch what they work on. Say, i have written some code. Now i might not feel confident in pushing it to production, because another developer has been in charge of managing the production environment, including the vps/droplet or how we build or something similar. By building a workflow, that essentially captures all the knowledge required, to safely deliver my new work to the production environment, or whatever needs to be done with it, i can now deliver my work to production myself, without causing errors, or disturbing the developer, who initially setup the environment i want to use. This way, both of us are more productive. \\

These are examples of how a DevOps mindset, and DevOps tooling, has allowed each developer to take more independent ownership of their work, and streamline our co-operation. 